\newpage

\section{Обзор существующих математических моделей и методов их решения} 

\subsection{Существующие математические модели технологического процесса}

В первой главе выпускной квалификационной работы бакалавра необходимо привести литературный обзор существующих моделей выбранной Вами технической, экономической и т.п. системы или технологического процесса. Здесь же дается описание предметной области применения выбранной системы и необходимые для полного дальнейшего понимания определения и понятия.


\subsection{Оформления списка библиографических источников}

При обзоре литературных источников необходимо делать на них затекстовые ссылки согласно дейтсвующему в РФ  \href{http://www.ivran.ru/attachments/552_P_7_0_5_-2008.pdf}{ГОСТ P 7.0.5-2008 <<Библиографическая ссылка. Общие требования и правила составления>>}. При написании выпускной работы рекомендуется использовать пакет BibTeX.

В списке источников данного файла приведен пример библиографической ссылки на:
\begin{itemize}
	\item книгу одного автора \cite{Krivulin2009},
	\item книгу нескольких авторов \cite{Barichev2011},
	\item книгу без указания автора \cite{BibOp91},
	\item журнальную статью \cite{Melikov92},
	\item тезисы в материалах конференции \cite{Ponomarenko88},
	\item стандарт \cite{GOST7052008},
	\item диссертацию и автореферат -- \cite{KIA95-default} и \cite{KIA95-autoref} соответственно.
\end{itemize}

\subsection{Теоремы и формулы}

Присутствующие в тексте работы теоремы, предложения, утверждения и леммы, а также различной сложности формулы рекомендуется оформлять пользуясь документом по следующей \href{http://mcc-conf.ru/d/math_in_latex.pdf}{ссылке}. 

\subsection{Постановка задач исследования}

Первая глава обычно заканчивается формальной постановкой основной задачи исследования и частных подзадач.


\newpage

\section{Математическая модель процесса, оптимизация процесса, управление процессом}

Во второй главе следует отразить рассматриваемую (анализируемую) Вами модель описанной выше системы или процесса. Глава должна содержать несколько пунктов в соотвествиии с поставленными выше задачами исследования.


\begin{theorem} \label{10}
	Формулировка теоремы, содержащая математические утверждения:
	\begin{equation*}
	x^2 + y^2 = z^2.
	\end{equation*}
\end{theorem}

\begin{proof}
	Доказательство приведенной теоремы. \qedhere
\end{proof}
	
Пример ссылки на теорему \ref{10}.


\newpage

\section{Программное обеспечение для решение поставленной задачи}

Третья глава выпускной квалификационной работы посвящена разработке программного обеспечения (программного комплекса) для решения заявленной задачи. Программа должна соответствовать описанным в главе 2 моделям и методам и отличаться оригинальностью.

\newpage

\section{Численное решение поставленной задачи}

В отличае от главы 2, в которой приводилось аналитическое решение рассматриваемой задачи, глава 4 должна содержать численное решение задачи с использованием разработанного и представленного в главе 3 программного обеспечения. Глава 4 может содержать оценку адекватности модели и метода ее решения, а также практические рекомендации специалистам предметной области по решению задачи и оценке полученных результатов.

Все главы выпускной квалификационной работы могут включать в графические элементы. Примеры рисунков и таблиц представлены в приложениях 1 и 2. 





