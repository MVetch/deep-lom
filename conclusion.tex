\newpage

\section*{Заключение} \addcontentsline{toc}{section}{Заключение}

Таким образом, была разработана программа, способная рассчитывать необходимые характеристики, указанные в постановке задачи. С помощью этой программы оператор стана может заранее смоделировать выходные параметры металлов после проката и принять правильное решение по настройке оборудования заранее. Тем самым  повышается качество конечного продукта, а так же предприятие, использующее эту программу может сэкономить ресурсы.

В первой главе были рассмотрены принципы горячей прокатки. Так же был рассмотрен существующий механизм процесса на ПАО <<НЛМК>> и поставлена задача.

Во второй главе задача была рассмотрена, переформулирована для лучшего понимания и переведена на математичекий язык. Была составлена математическая модель распределения тепла в глубину полосы и  рабочего слоя валка. Эта модель была представлена в виде аппроксимации для возможности ее запрограммировать и решить полученные уравнения численно.

В третьей главе была представлена программная реализация и руководство пользования программой.