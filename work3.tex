\section{Программное обеспечение для решение поставленной задачи}

\subsection{Введение}

\subsubsection{Область применения}
\subsubsection{Краткое описание возможностей}
\subsubsection{Уровень подготовки пользователя}

\subsection{Назначение и условия применения}

\subsection{Подготовка к работе}
\subsubsection{Состав и содержание дистрибутивного носителя данных}
\subsubsection{Порядок загрузки данных и программ}
\subsubsection{Порядок проверки работоспособности}

\subsection{Описание операций}
\subsubsection{Выполняемые функции и задачи}
\begin{longtable}{|p{3cm}|p{4cm}|p{8cm}|}
\hline
Функции & Задачи & Описание\\
\hline
 &   & \\
\hline
\end{longtable}

\subsection{Аварийные ситуации}
В случае возникновения ошибок при работе ИАС ППОС, не описанных ниже в данном разделе, необходимо обращаться к ответственному Администратору ИАС ППОС.
\begin{longtable}{|p{3cm}|p{3cm}|p{3cm}|p{6cm}|}
\hline
Класс ошибки & Ошибка & Описание ошибки & Требуемые действия пользователя при возникновении ошибки\\
\hline
 &   &  & \\
\hline
\end{longtable}

\subsection{Рекомендации по освоению}
В качестве контрольного примера рекомендуется выполнить операции, описанные в п. 4.2. настоящего документа.

\newpage