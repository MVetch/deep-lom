\section{Программное обеспечение для решения поставленной задачи}

\subsection{Введение}

\subsubsection{Область применения}
Программный продукт применяется для расчета основных характеристик при работе с процессом горячей прокатки. Получаемые данные необходимы для оценки изношенности рабочих валков, а так же для оценки уровня нагрева рабочего слоя. Информация о нагреве нужна для оптимизации процесса охлаждения валка.

\subsubsection{Краткое описание возможностей}
Программа может при помощи решения дифференициальных уравнений теплопроводности в частных производных расчитывать температурное распределение в полосе и рабочем слое валка. Для более точного расчета используются физические законы, описывающие энергосиловые характеристики процесса, с помощью которых программа рассчитывает тепловое распределение более точно. Вся полученная информация выводится в понятных интерактивных графиках для наглядного отображения получившихся параметров горячей прокатки.

\subsubsection{Уровень подготовки пользователя}
Чтобы использовать данную программу необходимо понимать предметную область. А именно понимать, как физически устроен процесс горячей прокатки. Такими знаниями обычно обладают операторы оборудования, управляющие процессом. Тем не менее, необходимым уровенем знаний обладают не только люди этой профессии и, при должном уровне доступа к этой программе, могут пользоваться и другие.

\subsection{Назначение и условия применения}

\subsection{Подготовка к работе}
\subsubsection{Состав и содержание дистрибутивного носителя данных}

\subsubsection{Порядок загрузки данных и программ}
\subsubsection{Порядок проверки работоспособности}

\subsection{Описание операций}
\subsubsection{Выполняемые функции и задачи}
\begin{longtable}{|p{3cm}|p{4cm}|p{8cm}|}
\hline
Функции & Задачи & Описание\\
\hline
Настройка параметров & Задание входных данных для расчета всех характеристик & Перед началом работы необходимо убедиться, что настройки оборудования соответствуют значениям, введенным в программе для расчета необходимых величин. В настройках задаются физические параметры металлов полосы и валка (коэффициент теплопроводности, плотность материалов и теплоемкость), параметры процесса прокатки для конкретной клети (радиус валка и толщины полосы на входе в очаг деформации и на выходе из него), а так же параметры дискретизации для решения задачи. Все настройки сохраняются в файл для дальнейшего их чтения. Сохранение настроек в файл позволяет удобно переносить с одного носителя на другой.\\
\hline
Основной расчет & Расчет всех необходимых характеристик процесса горячей прокатки & Основная функция программы. Производит расчет с заданными параметрами, считанными из файла настроек (см. Настройка параметров). В виде результата выводится таблица с распределением температур. Для анализа результатов выодятся графики энергосиловых характеристик и график распределения температуры на выходе из очага деформации в полосе и в рабочем слое валка.\\
\hline
\end{longtable}

\subsubsection{Описание исполняемых функций}

1) Настройка параметров.

Для выполнения этой функции необходимо выполнить следующие действия. На главном экране нажать на кнопку <<Открыть настройки>>.
\blockschema{7}{Кнопка для открытия окна с настройками}

Появится дополнительное окно, в котором необходимо указать параметры, описанные в п. 3.4.1.
\fig{8}{Окно настроек}

2) Основной расчет.

Чтобы выполнить эту функцию необходимо на главном экране нажать на кнопку <<Произвести расчет>>.
\blockschema{7}{Кнопка для основного расчета}

В течение нескольких секунд будет производиться расчет и вывод полученной информации на экран. Необходимо немного подождать завершения операции. В результате на экране появятся графики изменения самых важных величин (теплового потока, нормального и касательного напряжений, предела текучести и сопротивления деформации) по длине очага и график распределения тепла на выходе из очага. И, что самое главное, разностную сетку со значениями температуры в рабочем слое валка и в полосе.
\blockschema{9}{Главное окно после расчетов}

\subsection{Аварийные ситуации}
В случае возникновения ошибок при работе с программой, не описанных ниже в данном разделе, необходимо обращаться к ответственному администратору.
\begin{longtable}{|p{4cm}|p{4cm}|p{8cm}|}
\hline
Ошибка & Описание ошибки & Требуемые действия пользователя при возникновении ошибки\\
\hline
 &  &  & \\
\hline
\end{longtable}

\subsection{Рекомендации по освоению}
В качестве контрольного примера рекомендуется выполнить операции, описанные в п. 3.4.2. настоящего документа.

\newpage