\newpage

\section{Обзор процесса горячей прокатки и постановка задачи} 

\subsection{Процесс горячей прокатки}

Процесс пластической деформации металла между двумя или несколькими вращающимися рабочими валками называется прокаткой \cite{Tselikov}.

Прокатка осуществляется разными способами, которые различаются:
\begin{itemize}
\item направлением обработки (продольная, поперечная и винтовая
прокатка);
\item режимом работы станов (непрерывная и реверсивная прокатка);
\item состоянием металла (горячая, теплая, холодная прокатка);
\item формой изделия (лист, сплошной или пустотелый профиль).
\end{itemize}

Рабочие валки могут быть с гладкой бочкой или с нарезанными калибрами. Если оси валков параллельны и лежат в одной плоскости, валки имеют одинаковые диаметры и вращаются в разные стороны с одинаковыми окружными скоростями, прокатываемый металл однороден по своим механическим свойствам и на него действуют только силы от валков, то такой процесс прокатки называется \textit{простым}.

Пространство, ограниченное сверху и снизу дугами захвата (рис. 1), боковыми гранями полосы и плоскостями входа и выхода металла из валков, называется \textit{геометрическим очагом (зоной) деформации} (рис. 2).

\blockschema{2}{Схема очага деформации при прокатке}

\textit{Фактический очаг} деформации больше геометрического и включает в себя внеконтактные зоны, а также \textit{зоны упругой деформации}.

Разность  вцысот полосы при входе и выходе из валков называют \textit{абсолютным обжатием}: $\Delta h = h_0 - h_1$, разность между конечной и начально шириной полосы - \textit{уширением}: $\Delta B = B_1 - B_0$.

Дугу, по которой валок соприкасается с металлом, называют дугой захвата, а горизонтальную проекцию этой дуги $l$ принимают за длину геометрического очага деформации.

\textit{Углом захвата} $\alpha$ называют центральный угол, соответствующий дуге захвата, и находят по уравнению $$\cos \alpha = 1 - \frac{\Delta h}{D}.$$

При небольших углах захвата ($\alpha = 10 - 15^\circ$) можно считать, что $\alpha \approx \sin \alpha$, и тогда $$\alpha \approx \sqrt{\frac{\Delta h}{R}}.$$

\fig{3}{Схема очага деформации при прокатке с обозначениями}

\subsection{Технология производства ПАО <<НЛМК>>}

Новолипецкий металлургический комбинат является предприятием с полным металлургическим циклом, а это значит, что на промышленной площадке комбината располагаются все производства, необходимые для того, чтобы железная руда, пройдя все технологические этапы, превратилась в конечный металлургический продукт – холоднокатаный прокат, в том числе с покрытиями.
Общая схема производства включает следующие переделы: 

\begin{itemize}
\item Агломерационное производство с четырьмя агломашинами;
\item Коксохимическое производство с четырьмя батареями, оборудованными установками беспылевой выдачи кокса;
\item Доменное производство, представленное двумя доменными цехами с шестью доменными печами;
\item Сталеплавильное производство, представленное двумя конвертерными цехами, в состав которых входят шесть конвертеров и девять УНРС.
\item Прокатное производство, представленное цехом горячего проката с непрерывным широкополосным станом горячей прокатки 2000 и тремя цехами холодной прокатки, в состав которых входят два двадцативалковых стана, два реверсивных стана, один непрерывный стан, три дрессировочных стана, один полностью непрерывный стан <<бесконечной прокатки>>.
\end{itemize}

\blockschema{17}{Схема производства металла на ПАО <<НЛМК>>}

Прокатное производство представлено производством горячего проката (ПГП), производством холодного проката и покрытий (ПХПП), производством трансформаторной стали (ПТС) и производством динамной стали (ПДС). Сталь, прокатанная на стане <<2000>> ПГП (горячекатаный прокат), является товарной продукцией НЛМК третьего передела и служит заготовкой в производстве холоднокатаного проката. Тем важнее минимизировать издержки производства.

Производство горячекатаного проката на комбинате осуществляется на непрерывном широкополосном стане (НШС) <<2000>>. Производительность стана – около 5 млн 780 тысяч тонн проката в год. Длина технологической линии производства стальной горячекатаной полосы – 1,2 километра. Стан оснащен новейшими системами автомати-ческого управления, приводами всех основных механизмов, а также системами регули-рования и управления технологическим процессом.

На отводящем рольганге расположена автоматизированная система контроля качества поверхности полосы. На стане производят прокат толщиной 1,45-25,00 мм и шириной 900-1850 мм:

\begin{itemize}
\item товарный прокат из углеродистой и низколегированной стали на внутренний рынок и на экспорт;
\item прокат для дальнейшей холодной прокатки в ПХПП и ПДС (подкат) из углеродистой и низкоуглеродистой стали;
\item прокат из электротехнических марок стали (динамной и трансформаторной) для дальнейшей холодной прокатки в ПТС и ПДС.
\end{itemize}

Горячая прокатка начинается с предварительного разогрева слябов в методических нагревательных печах стана до температуры $1200-1250^\circ$С в течение 3–4 часов. Участок нагревательных печей имеет в своем составе пять нагревательных печей: две толкательные и три новые с шагающими балками. Нагрев в новых печах производится от математической модели. Печи отапливаются смешанным природно-доменным газом. Затем разогретые слябы выдаются на рольганг стана и транспортируются к черновой группе клетей. Черновая группа клетей состоит из:

\begin{itemize}
\item чернового вертикального окалиноломателя, который при помощи двух вертикально расположенных валков разрушает окалину с поверхности сляба;
\item реверсивной двухвалковой клети № 1;
\item четырех последовательно расположенных универсальных четырехвалковых клетей № 2–5.
\end{itemize}

\fig{18}{Универсальная четырехвалковая клеть черновой группы}

В черновой группе сляб проходит, так называемую, черновую (начальную) обработку, прокатываясь последовательно в каждой клети до нужной промежуточной толщины, в зависимости от конечной толщины проката. Для удаления окалины в линии стана установлены специальные приспособления (гидросбивы), которые струей воды (давлением 12,0–16,0 МПа) очищают поверхность металла. Из черновой группы клетей прокат (раскат) транспортируется по промежуточному рольгангу к чистовой группе клетей.

Чистовая группа стана состоит из:

\begin{itemize}
\item летучих ножниц для обрезки переднего и заднего концов раската;
\item чистового двухвалкового окалиноломателя для разрушения окалины, которая образуется при окислении металла на воздухе во время транспортировки раската по промежуточному рольгангу;
\item семи последовательно расположенных четырехвалковых клетей.
\end{itemize}

Все клети чистовой группы оборудованы гидронажимными устройствами. Клети № 7-12 оснащены современными устройствами осевой сдвижки и системой противоизгиба рабочих валков для регулирования поперечной разнотолщинности прокатываемых полос. В чистовой группе клетей производят чистовую, или, другими словами, окончательную прокатку до конечной (заданной) толщины полосы. После выхода из последней клети стана полоса транспортируется по отводящему рольгангу, где для обеспечения необходимых механических свойств металла и соблюдения температурного режима смотки охлаждается водой из установки ускоренного охлаждения (душирования) полосы, и далее сматывается в рулоны на моталках (3 гидравлические моталки).

\fig{19}{Чистовая группа клетей стана <<2000>>}

Смотанные рулоны обвязывают по образующей на автоматизированных машинах обвязки и в зависимости от назначения, по конвейеру направляют:
\begin{itemize}
\item в отделочное отделение для обработки или порезки на листы (полосы) на агрегатах резки с последующей отгрузкой потребителям – товарный прокат;
\item в отделочное отделение для последующей отгрузки железнодорожным транспортом в ПТС и ПДС – подкат для дальнейшей холодной прокатки;
\item в ПХПП – подкат для дальнейшей обработки.
\end{itemize}

\subsection{Постановка задачи}

В условиях невозможности модельно узнать энергосиловые и температурные значения в процессе горячей прокатки в очаге деформации, большое количество материала используется не для готового продукта, а лишь для тестов. Поэтому необходимо разработать программу, способную рассчитывать нужные характеристики исходя из параметров прокатки.

Необходимо, как результат, получить значение в зоне очага деформации следующих характеристик:
\begin{itemize}
\item величину теплового потока;
\item нормальное давление, оказываемое на полосу;
\item сопротивление деформации;
\item касательное давление и предел текучести;
\item распределение температуры.
\end{itemize}

\newpage