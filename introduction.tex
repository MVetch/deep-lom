\section*{Введение}
\addcontentsline{toc}{section}{Введение}

На металлургических предприятиях, в частности ПАО <<НЛМК>>, стоит проблема разработки новых технологий для горячей прокатки металла. Связано это с тем, что на данный момент нет средства для того, чтобы модельно проэкспериментировать с различными настройками оборудования и приходится разрабатывать новые методы прокатки эмпирически. Очевидно, при неудачных попытках огромное количество металла уходит на утилизацию. 

Чтобы избежать колоссальных материальных потерь, необходим способ, при котором можно было бы моделировать поведение физических свойств материалов и заранее определять настройки оборудования, при которых в результате прокатки не получится бракованная партия, а значит минимизировать потенциальные потери. Потери также большие из-за того, что часто необходимо не создавать, а оптимизировать существующие методы прокатки. Не сделав это, можно потерять довольно большое количество денег из-за низкой цены, которая в свою очередь будет зависеть от качества конечного продукта. К тому же можно потерять клиентуру, которая из-за слишком низкого качества металла начнет обращаться к другим поставщикам, даже несмотря на то, что они могут находится далеко и будут появляться дополнительные издержки от транспортировки в жертву лучшего качества.

В данной работе предлагается такой способ, а именно программное обеспечение, расчитывающее физические характеристики процесса прокатки. С его помощью оператор может оценить износ материалов при различных настройках оборудования. Так же информация, выводимая представленным ПО, полезна при оценке качества конечного продукта, что может способствовать более высокой его цене на рынке.
\newpage
